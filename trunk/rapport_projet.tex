\documentclass{article}

\usepackage[utf8]{inputenc}
\usepackage[T1]{fontenc}
\usepackage[francais]{babel}

\title{Rapport projet de programmation 2016}

\author{Groupe A1\_2}
\begin{document}
\maketitle
\tableofcontents

\section{Le projet}%dans cette section ne faire qu'une petite description seulemnt
%ici faut parler du projet rush hour
% dans quel but
%ce qu'il doit faire
%dans le cadre de la v1
%et dans le cadre de la v2
%et le solveur
\subsection{Première version}
\subsection{Deuxième version}
%le caprice du client
\subsection{Le solveur}


\section{Notre conception du projet}
%décrire notre projet
\subsection{Architecture du projet}
%décrire l'architecture du projet
%que décrit chaque dossier
\subsection{Les modules}
%donner tous les modules que l'on a fait et a quoi il serve
%ceux pour game piece affichage et list


\section{Les tests}%pour Gautier tu sera content comme ça
%decrit tes tests parle de ceux de la V1 et de la V2 ainsi que les derniers de la list chaine seulement car tu as seulement ci ceux la
%explique pourquoi maintenant tu mets des phrases avant chaque action pour dire ce que tu fais à chaque moment
%en gros parle de l'evolution de ton écriture de code
%tu peux sur la fin parler de la tab de hash mais pas trop vu que tu ne l'a pas commit tu en reparlera après

\section{Le solveur}
%on peut parler ici de la premiere approche que l'on a eu: table de hash et list
\subsection{Première approche}
%je laisse le soin a ALexis de remplir cette partie
\subsection{Deuxième approche}
%la je vais écrire cette partie


\section{Analyse mémoire}

\subsection{Valgrind}
%faire une analyse fuite mémoire sur tout notre projet et mettre en annexe capture d'écran
%y a t-il des problèmes de free et tout ça

\subsection{Couverture de code}
%pareil pour ça

\section{Organisation}

\subsection{Répartition des tâches}
%ici faut parler de qui a fait quoi je vous laisse remplir vous saurez a qui correspond chaque partie
\subsubsection{Module piece}
\subsubsection{Module game}
\subsubsection{Module affichage}
\subsubsection{Le solveur}
\subsubsection{Les structures de données}
\subsubsection{Les tests}

\subsection{Répartition dans le temps et en espace}
%ici il faut estimer le temps que l'on a passé sur le projet et estimer si la repartion en charge de travail était bien répartie sur tous les membres de groupe
%cette partie est censée emmenée sur les difficultés

\section{Les difficultés rencontrées}
%mettre toutes les difficultés que vous avez rencontrés
\subsection{Au niveau de la V1}
\subsection{Au niveau de la V2}
\subsection{Au niveau du solveur}

\section{Pour aller plus loin}
\subsection{Analyse des problèmes rencontrés}
%est-ce qu'il ont été facile a résoudre?
\subsection{Les améliorations à apporter au projet}
%qu'auriez vous envie de refaire?
\subsection{Ce que nous à apporter le projet}
\end{document}
